%% Generated by Sphinx.
\def\sphinxdocclass{report}
\documentclass[letterpaper,10pt,french]{sphinxmanual}
\ifdefined\pdfpxdimen
   \let\sphinxpxdimen\pdfpxdimen\else\newdimen\sphinxpxdimen
\fi \sphinxpxdimen=.75bp\relax
\ifdefined\pdfimageresolution
    \pdfimageresolution= \numexpr \dimexpr1in\relax/\sphinxpxdimen\relax
\fi
%% let collapsible pdf bookmarks panel have high depth per default
\PassOptionsToPackage{bookmarksdepth=5}{hyperref}

\PassOptionsToPackage{warn}{textcomp}
\usepackage[utf8]{inputenc}
\ifdefined\DeclareUnicodeCharacter
% support both utf8 and utf8x syntaxes
  \ifdefined\DeclareUnicodeCharacterAsOptional
    \def\sphinxDUC#1{\DeclareUnicodeCharacter{"#1}}
  \else
    \let\sphinxDUC\DeclareUnicodeCharacter
  \fi
  \sphinxDUC{00A0}{\nobreakspace}
  \sphinxDUC{2500}{\sphinxunichar{2500}}
  \sphinxDUC{2502}{\sphinxunichar{2502}}
  \sphinxDUC{2514}{\sphinxunichar{2514}}
  \sphinxDUC{251C}{\sphinxunichar{251C}}
  \sphinxDUC{2572}{\textbackslash}
\fi
\usepackage{cmap}
\usepackage[T1]{fontenc}
\usepackage{amsmath,amssymb,amstext}
\usepackage{babel}



\usepackage{tgtermes}
\usepackage{tgheros}
\renewcommand{\ttdefault}{txtt}



\usepackage[Sonny]{fncychap}
\ChNameVar{\Large\normalfont\sffamily}
\ChTitleVar{\Large\normalfont\sffamily}
\usepackage{sphinx}

\fvset{fontsize=auto}
\usepackage{geometry}


% Include hyperref last.
\usepackage{hyperref}
% Fix anchor placement for figures with captions.
\usepackage{hypcap}% it must be loaded after hyperref.
% Set up styles of URL: it should be placed after hyperref.
\urlstyle{same}

\addto\captionsfrench{\renewcommand{\contentsname}{Contents:}}

\usepackage{sphinxmessages}
\setcounter{tocdepth}{1}



\title{Labyrinthes}
\date{mars 27, 2023}
\release{}
\author{Alex Scofield Teruel, Hugo Taile\sphinxhyphen{}Manikom, Rayan Lalaoui}
\newcommand{\sphinxlogo}{\vbox{}}
\renewcommand{\releasename}{}
\makeindex
\begin{document}

\ifdefined\shorthandoff
  \ifnum\catcode`\=\string=\active\shorthandoff{=}\fi
  \ifnum\catcode`\"=\active\shorthandoff{"}\fi
\fi

\pagestyle{empty}
\sphinxmaketitle
\pagestyle{plain}
\sphinxtableofcontents
\pagestyle{normal}
\phantomsection\label{\detokenize{index::doc}}


\sphinxstepscope


\chapter{src}
\label{\detokenize{modules:src}}\label{\detokenize{modules::doc}}
\sphinxstepscope


\section{src package}
\label{\detokenize{src:src-package}}\label{\detokenize{src::doc}}

\subsection{Submodules}
\label{\detokenize{src:submodules}}

\subsection{src.affichage module}
\label{\detokenize{src:src-affichage-module}}

\subsection{src.affichage\_docs module}
\label{\detokenize{src:src-affichage-docs-module}}

\subsection{src.algorithmes module}
\label{\detokenize{src:src-algorithmes-module}}

\subsection{src.algorithmes\_docs module}
\label{\detokenize{src:module-src.algorithmes_docs}}\label{\detokenize{src:src-algorithmes-docs-module}}\index{module@\spxentry{module}!src.algorithmes\_docs@\spxentry{src.algorithmes\_docs}}\index{src.algorithmes\_docs@\spxentry{src.algorithmes\_docs}!module@\spxentry{module}}
\sphinxAtStartPar
Algorithmes.
\index{construit\_matrice\_labyrinthes() (dans le module src.algorithmes\_docs)@\spxentry{construit\_matrice\_labyrinthes()}\spxextra{dans le module src.algorithmes\_docs}}

\begin{fulllineitems}
\phantomsection\label{\detokenize{src:src.algorithmes_docs.construit_matrice_labyrinthes}}
\pysigstartsignatures
\pysiglinewithargsret{\sphinxcode{\sphinxupquote{src.algorithmes\_docs.}}\sphinxbfcode{\sphinxupquote{construit\_matrice\_labyrinthes}}}{\emph{\DUrole{n}{taille}\DUrole{p}{:}\DUrole{w}{  }\DUrole{n}{tuple}}}{{ $\rightarrow$ list}}
\pysigstopsignatures
\sphinxAtStartPar
Construit tous les PseudoLabyrinthes d’une taille donnée.
\begin{quote}\begin{description}
\sphinxlineitem{Paramètres}
\sphinxAtStartPar
\sphinxstyleliteralstrong{\sphinxupquote{taille}} (\sphinxstyleliteralemphasis{\sphinxupquote{tuple}}) \textendash{} Taille des pseudo\sphinxhyphen{}labyrinthes à construire.

\sphinxlineitem{Renvoie}
\sphinxAtStartPar
Une liste contenant tous les PseudoLabyrinthes sous forme de matrice de murs valides de la taille donnée.

\sphinxlineitem{Type renvoyé}
\sphinxAtStartPar
list

\end{description}\end{quote}

\end{fulllineitems}

\index{construit\_pseudo\_labyrinthe\_vide() (dans le module src.algorithmes\_docs)@\spxentry{construit\_pseudo\_labyrinthe\_vide()}\spxextra{dans le module src.algorithmes\_docs}}

\begin{fulllineitems}
\phantomsection\label{\detokenize{src:src.algorithmes_docs.construit_pseudo_labyrinthe_vide}}
\pysigstartsignatures
\pysiglinewithargsret{\sphinxcode{\sphinxupquote{src.algorithmes\_docs.}}\sphinxbfcode{\sphinxupquote{construit\_pseudo\_labyrinthe\_vide}}}{\emph{\DUrole{n}{taille}\DUrole{p}{:}\DUrole{w}{  }\DUrole{n}{tuple}}}{}
\pysigstopsignatures
\sphinxAtStartPar
Construit un PseudoLabyrinthe dans lequel toutes les connexions sont faites.
\begin{quote}\begin{description}
\sphinxlineitem{Paramètres}
\sphinxAtStartPar
\sphinxstyleliteralstrong{\sphinxupquote{taille}} (\sphinxstyleliteralemphasis{\sphinxupquote{tuple}}) \textendash{} Tuple contenant la taille du PseudoLabyrinthe à construire.

\sphinxlineitem{Renvoie}
\sphinxAtStartPar
Un Pseudo\sphinxhyphen{}labyrinthe sans murs.

\sphinxlineitem{Type renvoyé}
\sphinxAtStartPar
{\hyperref[\detokenize{src:src.utilites.PseudoLabyrinthe}]{\sphinxcrossref{PseudoLabyrinthe}}}

\end{description}\end{quote}

\end{fulllineitems}

\index{construit\_random\_labyrinthe\_supprime() (dans le module src.algorithmes\_docs)@\spxentry{construit\_random\_labyrinthe\_supprime()}\spxextra{dans le module src.algorithmes\_docs}}

\begin{fulllineitems}
\phantomsection\label{\detokenize{src:src.algorithmes_docs.construit_random_labyrinthe_supprime}}
\pysigstartsignatures
\pysiglinewithargsret{\sphinxcode{\sphinxupquote{src.algorithmes\_docs.}}\sphinxbfcode{\sphinxupquote{construit\_random\_labyrinthe\_supprime}}}{\emph{\DUrole{n}{taille}\DUrole{p}{:}\DUrole{w}{  }\DUrole{n}{tuple}}}{{ $\rightarrow$ {\hyperref[\detokenize{src:src.utilites.PseudoLabyrinthe}]{\sphinxcrossref{PseudoLabyrinthe}}}}}
\pysigstopsignatures
\sphinxAtStartPar
Construit un labyrinthe random de taille donnée.
\begin{quote}\begin{description}
\sphinxlineitem{Paramètres}
\sphinxAtStartPar
\sphinxstyleliteralstrong{\sphinxupquote{taille}} (\sphinxstyleliteralemphasis{\sphinxupquote{tuple}}) \textendash{} Un tuple contenant la taille du labyrinthe à construire.

\sphinxlineitem{Renvoie}
\sphinxAtStartPar
Un labyrinthe random de cette taille.

\sphinxlineitem{Type renvoyé}
\sphinxAtStartPar
{\hyperref[\detokenize{src:src.utilites.PseudoLabyrinthe}]{\sphinxcrossref{PseudoLabyrinthe}}}

\end{description}\end{quote}

\end{fulllineitems}

\index{construit\_random\_pseudo\_labyrinthe\_ajoute() (dans le module src.algorithmes\_docs)@\spxentry{construit\_random\_pseudo\_labyrinthe\_ajoute()}\spxextra{dans le module src.algorithmes\_docs}}

\begin{fulllineitems}
\phantomsection\label{\detokenize{src:src.algorithmes_docs.construit_random_pseudo_labyrinthe_ajoute}}
\pysigstartsignatures
\pysiglinewithargsret{\sphinxcode{\sphinxupquote{src.algorithmes\_docs.}}\sphinxbfcode{\sphinxupquote{construit\_random\_pseudo\_labyrinthe\_ajoute}}}{\emph{\DUrole{n}{taille}\DUrole{p}{:}\DUrole{w}{  }\DUrole{n}{tuple}}}{{ $\rightarrow$ {\hyperref[\detokenize{src:src.utilites.PseudoLabyrinthe}]{\sphinxcrossref{PseudoLabyrinthe}}}}}
\pysigstopsignatures
\sphinxAtStartPar
Construit un Pseudolabyrinthe random de la taille donée en ajoutant des murs.
\begin{quote}\begin{description}
\sphinxlineitem{Paramètres}
\sphinxAtStartPar
\sphinxstyleliteralstrong{\sphinxupquote{taille}} (\sphinxstyleliteralemphasis{\sphinxupquote{tuple}}) \textendash{} Un tuple contenant la taille du labyrinthe à construire.

\sphinxlineitem{Type renvoyé}
\sphinxAtStartPar
PseudoLabyrinthe random de cette taille.

\end{description}\end{quote}

\end{fulllineitems}

\index{construit\_random\_pseudo\_labyrinthe\_supprime() (dans le module src.algorithmes\_docs)@\spxentry{construit\_random\_pseudo\_labyrinthe\_supprime()}\spxextra{dans le module src.algorithmes\_docs}}

\begin{fulllineitems}
\phantomsection\label{\detokenize{src:src.algorithmes_docs.construit_random_pseudo_labyrinthe_supprime}}
\pysigstartsignatures
\pysiglinewithargsret{\sphinxcode{\sphinxupquote{src.algorithmes\_docs.}}\sphinxbfcode{\sphinxupquote{construit\_random\_pseudo\_labyrinthe\_supprime}}}{\emph{\DUrole{n}{taille}\DUrole{p}{:}\DUrole{w}{  }\DUrole{n}{tuple}}}{{ $\rightarrow$ {\hyperref[\detokenize{src:src.utilites.PseudoLabyrinthe}]{\sphinxcrossref{PseudoLabyrinthe}}}}}
\pysigstopsignatures
\sphinxAtStartPar
Construit un Pseudo\sphinxhyphen{}labyrinthe random, de taille donnée.
\begin{quote}\begin{description}
\sphinxlineitem{Paramètres}
\sphinxAtStartPar
\sphinxstyleliteralstrong{\sphinxupquote{taille}} (\sphinxstyleliteralemphasis{\sphinxupquote{tuple}}) \textendash{} taille du PseudoLabyrinthe.

\sphinxlineitem{Renvoie}
\sphinxAtStartPar
pseudo\sphinxhyphen{}labyrinthe random de la taille voulue.

\sphinxlineitem{Type renvoyé}
\sphinxAtStartPar
{\hyperref[\detokenize{src:src.utilites.PseudoLabyrinthe}]{\sphinxcrossref{PseudoLabyrinthe}}}

\end{description}\end{quote}

\end{fulllineitems}

\index{filtre\_liste\_PseudoLabyrinthe() (dans le module src.algorithmes\_docs)@\spxentry{filtre\_liste\_PseudoLabyrinthe()}\spxextra{dans le module src.algorithmes\_docs}}

\begin{fulllineitems}
\phantomsection\label{\detokenize{src:src.algorithmes_docs.filtre_liste_PseudoLabyrinthe}}
\pysigstartsignatures
\pysiglinewithargsret{\sphinxcode{\sphinxupquote{src.algorithmes\_docs.}}\sphinxbfcode{\sphinxupquote{filtre\_liste\_PseudoLabyrinthe}}}{\emph{\DUrole{n}{pseudo\_labyrinthes}\DUrole{p}{:}\DUrole{w}{  }\DUrole{n}{list}}}{}
\pysigstopsignatures
\sphinxAtStartPar
Extrait les labyrinthes d’une liste de pseudo\sphinxhyphen{}labyrinthes.
\begin{quote}\begin{description}
\sphinxlineitem{Paramètres}
\sphinxAtStartPar
\sphinxstyleliteralstrong{\sphinxupquote{pseudo\sphinxhyphen{}labyrinthes}} (\sphinxstyleliteralemphasis{\sphinxupquote{list}}) \textendash{} Une liste de pseudo\sphinxhyphen{}labyrinthes

\sphinxlineitem{Renvoie}
\sphinxAtStartPar
Une liste contenant les labyrinthes de pseudo\_labyrinthes

\sphinxlineitem{Type renvoyé}
\sphinxAtStartPar
list

\end{description}\end{quote}

\end{fulllineitems}

\index{get\_Labyrinthes() (dans le module src.algorithmes\_docs)@\spxentry{get\_Labyrinthes()}\spxextra{dans le module src.algorithmes\_docs}}

\begin{fulllineitems}
\phantomsection\label{\detokenize{src:src.algorithmes_docs.get_Labyrinthes}}
\pysigstartsignatures
\pysiglinewithargsret{\sphinxcode{\sphinxupquote{src.algorithmes\_docs.}}\sphinxbfcode{\sphinxupquote{get\_Labyrinthes}}}{\emph{\DUrole{n}{taille}\DUrole{p}{:}\DUrole{w}{  }\DUrole{n}{tuple}}}{}
\pysigstopsignatures
\sphinxAtStartPar
Consruit tous les labyrinthes pour une taille donnée.
\begin{quote}\begin{description}
\sphinxlineitem{Paramètres}
\sphinxAtStartPar
\sphinxstyleliteralstrong{\sphinxupquote{taille}} (\sphinxstyleliteralemphasis{\sphinxupquote{tuple}}) \textendash{} Taille des labyrinthes à construire

\sphinxlineitem{Renvoie}
\sphinxAtStartPar
Une liste contenant tous les labyrinthes de taille taille.

\sphinxlineitem{Type renvoyé}
\sphinxAtStartPar
list

\end{description}\end{quote}

\end{fulllineitems}

\index{get\_PseudoLabyrinthes() (dans le module src.algorithmes\_docs)@\spxentry{get\_PseudoLabyrinthes()}\spxextra{dans le module src.algorithmes\_docs}}

\begin{fulllineitems}
\phantomsection\label{\detokenize{src:src.algorithmes_docs.get_PseudoLabyrinthes}}
\pysigstartsignatures
\pysiglinewithargsret{\sphinxcode{\sphinxupquote{src.algorithmes\_docs.}}\sphinxbfcode{\sphinxupquote{get\_PseudoLabyrinthes}}}{\emph{\DUrole{n}{taille}\DUrole{p}{:}\DUrole{w}{  }\DUrole{n}{tuple}}}{}
\pysigstopsignatures
\sphinxAtStartPar
Transforme les éléments de la matrice de murs de construit\_matrice\_labyrinthe en pseudo\sphinxhyphen{}labyrinthes pour une taille donnée.
\begin{quote}\begin{description}
\sphinxlineitem{Paramètres}
\sphinxAtStartPar
\sphinxstyleliteralstrong{\sphinxupquote{taille}} (\sphinxstyleliteralemphasis{\sphinxupquote{tuple}}) \textendash{} Taille des pseudo\sphinxhyphen{}labyrinthes à construire.

\sphinxlineitem{Renvoie}
\sphinxAtStartPar
liste des pseudo\sphinxhyphen{}labyrinthes construits

\sphinxlineitem{Type renvoyé}
\sphinxAtStartPar
list

\end{description}\end{quote}

\end{fulllineitems}

\index{murs\_to\_PseudoLabyrinthe() (dans le module src.algorithmes\_docs)@\spxentry{murs\_to\_PseudoLabyrinthe()}\spxextra{dans le module src.algorithmes\_docs}}

\begin{fulllineitems}
\phantomsection\label{\detokenize{src:src.algorithmes_docs.murs_to_PseudoLabyrinthe}}
\pysigstartsignatures
\pysiglinewithargsret{\sphinxcode{\sphinxupquote{src.algorithmes\_docs.}}\sphinxbfcode{\sphinxupquote{murs\_to\_PseudoLabyrinthe}}}{\emph{\DUrole{n}{collection\_murs}\DUrole{p}{:}\DUrole{w}{  }\DUrole{n}{tuple}}, \emph{\DUrole{n}{taille}\DUrole{p}{:}\DUrole{w}{  }\DUrole{n}{tuple}}}{}
\pysigstopsignatures
\sphinxAtStartPar
Converti une collection de murs en des pseudo\sphinxhyphen{}labyrinthes.
\begin{quote}\begin{description}
\sphinxlineitem{Paramètres}\begin{itemize}
\item {} 
\sphinxAtStartPar
\sphinxstyleliteralstrong{\sphinxupquote{collection\_murs}} (\sphinxstyleliteralemphasis{\sphinxupquote{tuple}}) \textendash{} Un tuple contenant les murs à ajouter.

\item {} 
\sphinxAtStartPar
\sphinxstyleliteralstrong{\sphinxupquote{taille}} (\sphinxstyleliteralemphasis{\sphinxupquote{tuple}}) \textendash{} taille du pseudo\sphinxhyphen{}labyrinthe.

\end{itemize}

\sphinxlineitem{Renvoie}
\sphinxAtStartPar
Le pseudo\sphinxhyphen{}labyrinthe construit à partir de collection\_murs.

\sphinxlineitem{Type renvoyé}
\sphinxAtStartPar
{\hyperref[\detokenize{src:src.utilites.PseudoLabyrinthe}]{\sphinxcrossref{PseudoLabyrinthe}}}

\end{description}\end{quote}

\end{fulllineitems}

\index{nb\_murs() (dans le module src.algorithmes\_docs)@\spxentry{nb\_murs()}\spxextra{dans le module src.algorithmes\_docs}}

\begin{fulllineitems}
\phantomsection\label{\detokenize{src:src.algorithmes_docs.nb_murs}}
\pysigstartsignatures
\pysiglinewithargsret{\sphinxcode{\sphinxupquote{src.algorithmes\_docs.}}\sphinxbfcode{\sphinxupquote{nb\_murs}}}{\emph{\DUrole{n}{pl}\DUrole{p}{:}\DUrole{w}{  }\DUrole{n}{{\hyperref[\detokenize{src:src.utilites.PseudoLabyrinthe}]{\sphinxcrossref{PseudoLabyrinthe}}}}}}{{ $\rightarrow$ int}}
\pysigstopsignatures
\sphinxAtStartPar
Compte le nombre de murs dans un Pseudo\sphinxhyphen{}Labyrinthe, enceinte exclue
\begin{quote}\begin{description}
\sphinxlineitem{Paramètres}
\sphinxAtStartPar
\sphinxstyleliteralstrong{\sphinxupquote{pl}} ({\hyperref[\detokenize{src:src.utilites.PseudoLabyrinthe}]{\sphinxcrossref{\sphinxstyleliteralemphasis{\sphinxupquote{PseudoLabyrinthe}}}}}) \textendash{} Pseudo\sphinxhyphen{}labyrinthe dont on souhaite compter les murs

\sphinxlineitem{Renvoie}
\sphinxAtStartPar
Le nombre de murs de pl

\sphinxlineitem{Type renvoyé}
\sphinxAtStartPar
int

\end{description}\end{quote}

\end{fulllineitems}

\index{verifie\_connexite() (dans le module src.algorithmes\_docs)@\spxentry{verifie\_connexite()}\spxextra{dans le module src.algorithmes\_docs}}

\begin{fulllineitems}
\phantomsection\label{\detokenize{src:src.algorithmes_docs.verifie_connexite}}
\pysigstartsignatures
\pysiglinewithargsret{\sphinxcode{\sphinxupquote{src.algorithmes\_docs.}}\sphinxbfcode{\sphinxupquote{verifie\_connexite}}}{\emph{\DUrole{n}{pseudo\_labyrinthe}\DUrole{p}{:}\DUrole{w}{  }\DUrole{n}{{\hyperref[\detokenize{src:src.utilites.PseudoLabyrinthe}]{\sphinxcrossref{PseudoLabyrinthe}}}}}}{{ $\rightarrow$ bool}}
\pysigstopsignatures
\sphinxAtStartPar
Application de l’algorithme bfs pour vérifier si un Pseudo\sphinxhyphen{}Labyrinthe est connexe.
\begin{quote}\begin{description}
\sphinxlineitem{Paramètres}
\sphinxAtStartPar
\sphinxstyleliteralstrong{\sphinxupquote{pseudo\_labyrinthe}} ({\hyperref[\detokenize{src:src.utilites.PseudoLabyrinthe}]{\sphinxcrossref{\sphinxstyleliteralemphasis{\sphinxupquote{PseudoLabyrinthe}}}}}) \textendash{} Labyrinthe à vérifier.

\sphinxlineitem{Renvoie}
\sphinxAtStartPar
True si le labyrinthe est connexe, False sinon.

\sphinxlineitem{Type renvoyé}
\sphinxAtStartPar
bool

\end{description}\end{quote}

\end{fulllineitems}

\index{verifie\_labyrinthe() (dans le module src.algorithmes\_docs)@\spxentry{verifie\_labyrinthe()}\spxextra{dans le module src.algorithmes\_docs}}

\begin{fulllineitems}
\phantomsection\label{\detokenize{src:src.algorithmes_docs.verifie_labyrinthe}}
\pysigstartsignatures
\pysiglinewithargsret{\sphinxcode{\sphinxupquote{src.algorithmes\_docs.}}\sphinxbfcode{\sphinxupquote{verifie\_labyrinthe}}}{\emph{\DUrole{n}{pl}\DUrole{p}{:}\DUrole{w}{  }\DUrole{n}{{\hyperref[\detokenize{src:src.utilites.PseudoLabyrinthe}]{\sphinxcrossref{PseudoLabyrinthe}}}}}}{{ $\rightarrow$ bool}}
\pysigstopsignatures
\sphinxAtStartPar
Vérifie si un PseudoLabyrinthe est un Labyrinthe.
\begin{quote}\begin{description}
\sphinxlineitem{Paramètres}
\sphinxAtStartPar
\sphinxstyleliteralstrong{\sphinxupquote{pl}} ({\hyperref[\detokenize{src:src.utilites.PseudoLabyrinthe}]{\sphinxcrossref{\sphinxstyleliteralemphasis{\sphinxupquote{PseudoLabyrinthe}}}}}) \textendash{} 

\sphinxlineitem{Renvoie}
\sphinxAtStartPar
True si c’est un labyrinthe, False si non.

\sphinxlineitem{Type renvoyé}
\sphinxAtStartPar
bool

\end{description}\end{quote}

\end{fulllineitems}



\subsection{src.gui module}
\label{\detokenize{src:src-gui-module}}

\subsection{src.test\_algorithmes module}
\label{\detokenize{src:src-test-algorithmes-module}}

\subsection{src.test\_utilites module}
\label{\detokenize{src:src-test-utilites-module}}

\subsection{src.utilites module}
\label{\detokenize{src:module-src.utilites}}\label{\detokenize{src:src-utilites-module}}\index{module@\spxentry{module}!src.utilites@\spxentry{src.utilites}}\index{src.utilites@\spxentry{src.utilites}!module@\spxentry{module}}
\sphinxAtStartPar
Construction de structures de données utiles.
\index{Noeud (classe dans src.utilites)@\spxentry{Noeud}\spxextra{classe dans src.utilites}}

\begin{fulllineitems}
\phantomsection\label{\detokenize{src:src.utilites.Noeud}}
\pysigstartsignatures
\pysiglinewithargsret{\sphinxbfcode{\sphinxupquote{class\DUrole{w}{  }}}\sphinxcode{\sphinxupquote{src.utilites.}}\sphinxbfcode{\sphinxupquote{Noeud}}}{\emph{\DUrole{n}{id}\DUrole{p}{:}\DUrole{w}{  }\DUrole{n}{tuple}}, \emph{\DUrole{n}{connexions}\DUrole{o}{=}\DUrole{default_value}{{[}{]}}}}{}
\pysigstopsignatures
\sphinxAtStartPar
Bases : \sphinxcode{\sphinxupquote{object}}

\sphinxAtStartPar
Un Noeud est un objet qui se correspond à une case d’un PseudoLabyrinthe.
\begin{description}
\sphinxlineitem{id}{[}tuple{]}
\sphinxAtStartPar
Contient la position du noeud.

\sphinxlineitem{connexions}{[}liste de noeuds{]}
\sphinxAtStartPar
Noeuds connexes au Noeud courant

\end{description}
\index{\_\_id (attribut src.utilites.Noeud)@\spxentry{\_\_id}\spxextra{attribut src.utilites.Noeud}}

\begin{fulllineitems}
\phantomsection\label{\detokenize{src:src.utilites.Noeud.__id}}
\pysigstartsignatures
\pysigline{\sphinxbfcode{\sphinxupquote{\_\_id}}}
\pysigstopsignatures
\sphinxAtStartPar
Contient la position du noeud.
\begin{quote}\begin{description}
\sphinxlineitem{Type}
\sphinxAtStartPar
tuple

\end{description}\end{quote}

\end{fulllineitems}

\index{\_\_connexions (attribut src.utilites.Noeud)@\spxentry{\_\_connexions}\spxextra{attribut src.utilites.Noeud}}

\begin{fulllineitems}
\phantomsection\label{\detokenize{src:src.utilites.Noeud.__connexions}}
\pysigstartsignatures
\pysigline{\sphinxbfcode{\sphinxupquote{\_\_connexions}}}
\pysigstopsignatures
\sphinxAtStartPar
Noeuds connexes au Noeud courant
\begin{quote}\begin{description}
\sphinxlineitem{Type}
\sphinxAtStartPar
liste de noeuds

\end{description}\end{quote}

\end{fulllineitems}

\index{ajoute\_connexions() (méthode src.utilites.Noeud)@\spxentry{ajoute\_connexions()}\spxextra{méthode src.utilites.Noeud}}

\begin{fulllineitems}
\phantomsection\label{\detokenize{src:src.utilites.Noeud.ajoute_connexions}}
\pysigstartsignatures
\pysiglinewithargsret{\sphinxbfcode{\sphinxupquote{ajoute\_connexions}}}{\emph{\DUrole{o}{*}\DUrole{n}{args}}}{}
\pysigstopsignatures
\sphinxAtStartPar
Ajoute des connexions à self.\_\_connexions. Vérifie que les connexions sont valides.
\subsubsection*{Notes}

\sphinxAtStartPar
Dans la plupart des cas utiliser ajoute\_murs et ne pas ajoute\_connexions.
\begin{quote}\begin{description}
\sphinxlineitem{Paramètres}
\sphinxAtStartPar
\sphinxstyleliteralstrong{\sphinxupquote{*args}} (\sphinxstyleliteralemphasis{\sphinxupquote{list of Noeud}}) \textendash{} 

\end{description}\end{quote}

\end{fulllineitems}

\index{get\_connexions() (méthode src.utilites.Noeud)@\spxentry{get\_connexions()}\spxextra{méthode src.utilites.Noeud}}

\begin{fulllineitems}
\phantomsection\label{\detokenize{src:src.utilites.Noeud.get_connexions}}
\pysigstartsignatures
\pysiglinewithargsret{\sphinxbfcode{\sphinxupquote{get\_connexions}}}{}{{ $\rightarrow$ list}}
\pysigstopsignatures
\sphinxAtStartPar
Getter pour self.\_\_connexions.
\begin{quote}\begin{description}
\sphinxlineitem{Renvoie}
\sphinxAtStartPar
self.\_\_connexions

\sphinxlineitem{Type renvoyé}
\sphinxAtStartPar
list

\end{description}\end{quote}

\end{fulllineitems}

\index{get\_id() (méthode src.utilites.Noeud)@\spxentry{get\_id()}\spxextra{méthode src.utilites.Noeud}}

\begin{fulllineitems}
\phantomsection\label{\detokenize{src:src.utilites.Noeud.get_id}}
\pysigstartsignatures
\pysiglinewithargsret{\sphinxbfcode{\sphinxupquote{get\_id}}}{}{{ $\rightarrow$ tuple}}
\pysigstopsignatures
\sphinxAtStartPar
Getter pour self.\_\_id.
\begin{quote}\begin{description}
\sphinxlineitem{Renvoie}
\sphinxAtStartPar
self.\_\_id

\sphinxlineitem{Type renvoyé}
\sphinxAtStartPar
tuple

\end{description}\end{quote}

\end{fulllineitems}

\index{get\_voisins() (méthode src.utilites.Noeud)@\spxentry{get\_voisins()}\spxextra{méthode src.utilites.Noeud}}

\begin{fulllineitems}
\phantomsection\label{\detokenize{src:src.utilites.Noeud.get_voisins}}
\pysigstartsignatures
\pysiglinewithargsret{\sphinxbfcode{\sphinxupquote{get\_voisins}}}{\emph{\DUrole{n}{pseudo\_labyrinthe}\DUrole{p}{:}\DUrole{w}{  }\DUrole{n}{{\hyperref[\detokenize{src:src.utilites.PseudoLabyrinthe}]{\sphinxcrossref{PseudoLabyrinthe}}}}}}{{ $\rightarrow$ list}}
\pysigstopsignatures
\sphinxAtStartPar
Trouve les voisins du noeud dans le PseudoLabyrinthe donné comme paramètre.
\begin{quote}\begin{description}
\sphinxlineitem{Paramètres}
\sphinxAtStartPar
\sphinxstyleliteralstrong{\sphinxupquote{pseudo\_labyrinthe}} ({\hyperref[\detokenize{src:src.utilites.PseudoLabyrinthe}]{\sphinxcrossref{\sphinxstyleliteralemphasis{\sphinxupquote{PseudoLabyrinthe}}}}}) \textendash{} PseudoLabyrinthe où se trouve le Noeud.

\sphinxlineitem{Renvoie}
\sphinxAtStartPar
Contient les noeuds voisins du noeud en question dans pseudo\_labyrinthe.

\sphinxlineitem{Type renvoyé}
\sphinxAtStartPar
list

\end{description}\end{quote}

\end{fulllineitems}

\index{supprime\_connexions() (méthode src.utilites.Noeud)@\spxentry{supprime\_connexions()}\spxextra{méthode src.utilites.Noeud}}

\begin{fulllineitems}
\phantomsection\label{\detokenize{src:src.utilites.Noeud.supprime_connexions}}
\pysigstartsignatures
\pysiglinewithargsret{\sphinxbfcode{\sphinxupquote{supprime\_connexions}}}{\emph{\DUrole{o}{*}\DUrole{n}{args}}}{{ $\rightarrow$ None}}
\pysigstopsignatures
\sphinxAtStartPar
Supprime des connexions de self.\_\_connexions.

\sphinxAtStartPar
{\color{red}\bfseries{}*}args : Liste of Noeud

\end{fulllineitems}

\index{supprime\_connexions\_redoublantes() (méthode src.utilites.Noeud)@\spxentry{supprime\_connexions\_redoublantes()}\spxextra{méthode src.utilites.Noeud}}

\begin{fulllineitems}
\phantomsection\label{\detokenize{src:src.utilites.Noeud.supprime_connexions_redoublantes}}
\pysigstartsignatures
\pysiglinewithargsret{\sphinxbfcode{\sphinxupquote{supprime\_connexions\_redoublantes}}}{}{{ $\rightarrow$ None}}
\pysigstopsignatures
\sphinxAtStartPar
Supprime les connexions rédoublantes du Noeud.

\end{fulllineitems}


\end{fulllineitems}

\index{PseudoLabyrinthe (classe dans src.utilites)@\spxentry{PseudoLabyrinthe}\spxextra{classe dans src.utilites}}

\begin{fulllineitems}
\phantomsection\label{\detokenize{src:src.utilites.PseudoLabyrinthe}}
\pysigstartsignatures
\pysiglinewithargsret{\sphinxbfcode{\sphinxupquote{class\DUrole{w}{  }}}\sphinxcode{\sphinxupquote{src.utilites.}}\sphinxbfcode{\sphinxupquote{PseudoLabyrinthe}}}{\emph{\DUrole{n}{taille}\DUrole{p}{:}\DUrole{w}{  }\DUrole{n}{tuple}}}{}
\pysigstopsignatures
\sphinxAtStartPar
Bases : \sphinxcode{\sphinxupquote{object}}

\sphinxAtStartPar
Un PseudoLabyrinthe de taille m*n est une graphe avec m*n noeuds où chaque
noeud ne peut se connecter qu’à ses voisins.
\begin{quote}\begin{description}
\sphinxlineitem{Paramètres}
\sphinxAtStartPar
\sphinxstyleliteralstrong{\sphinxupquote{taille}} (\sphinxstyleliteralemphasis{\sphinxupquote{tuple}}) \textendash{} Taille du PseudoLabyrinthe à construir.

\end{description}\end{quote}
\index{taille (attribut src.utilites.PseudoLabyrinthe)@\spxentry{taille}\spxextra{attribut src.utilites.PseudoLabyrinthe}}

\begin{fulllineitems}
\phantomsection\label{\detokenize{src:src.utilites.PseudoLabyrinthe.taille}}
\pysigstartsignatures
\pysigline{\sphinxbfcode{\sphinxupquote{taille}}}
\pysigstopsignatures
\sphinxAtStartPar
Contient les dimensions du pseudo\_labyrinthes.
\begin{quote}\begin{description}
\sphinxlineitem{Type}
\sphinxAtStartPar
tuple

\end{description}\end{quote}

\end{fulllineitems}

\index{\_\_noeuds (attribut src.utilites.PseudoLabyrinthe)@\spxentry{\_\_noeuds}\spxextra{attribut src.utilites.PseudoLabyrinthe}}

\begin{fulllineitems}
\phantomsection\label{\detokenize{src:src.utilites.PseudoLabyrinthe.__noeuds}}
\pysigstartsignatures
\pysigline{\sphinxbfcode{\sphinxupquote{\_\_noeuds}}}
\pysigstopsignatures
\sphinxAtStartPar
Liste de Noeuds.
\begin{quote}\begin{description}
\sphinxlineitem{Type}
\sphinxAtStartPar
list of Noeud

\end{description}\end{quote}

\end{fulllineitems}

\index{ajoute\_murs() (méthode src.utilites.PseudoLabyrinthe)@\spxentry{ajoute\_murs()}\spxextra{méthode src.utilites.PseudoLabyrinthe}}

\begin{fulllineitems}
\phantomsection\label{\detokenize{src:src.utilites.PseudoLabyrinthe.ajoute_murs}}
\pysigstartsignatures
\pysiglinewithargsret{\sphinxbfcode{\sphinxupquote{ajoute\_murs}}}{\emph{\DUrole{o}{*}\DUrole{n}{args}}}{{ $\rightarrow$ None}}
\pysigstopsignatures
\sphinxAtStartPar
Procédure qui ajoute un mur entre les noeuds donnés comme paramètres deux par deux.
\begin{quote}\begin{description}
\sphinxlineitem{Paramètres}
\sphinxAtStartPar
\sphinxstyleliteralstrong{\sphinxupquote{*args}} (\sphinxstyleliteralemphasis{\sphinxupquote{tuple of Noeud}}) \textendash{} Tuples de Noeuds à connecter deux par deux.

\end{description}\end{quote}

\end{fulllineitems}

\index{bidirectionalise() (méthode src.utilites.PseudoLabyrinthe)@\spxentry{bidirectionalise()}\spxextra{méthode src.utilites.PseudoLabyrinthe}}

\begin{fulllineitems}
\phantomsection\label{\detokenize{src:src.utilites.PseudoLabyrinthe.bidirectionalise}}
\pysigstartsignatures
\pysiglinewithargsret{\sphinxbfcode{\sphinxupquote{bidirectionalise}}}{}{{ $\rightarrow$ None}}
\pysigstopsignatures
\sphinxAtStartPar
Procédure qui bidirectionalise toutes les connexions entre noeuds du PseudoLabyrinthe.
\subsubsection*{Notes}

\sphinxAtStartPar
Ne devrait pas être nécessaire avec l’implementation bidirectionelle.

\end{fulllineitems}

\index{construit() (méthode src.utilites.PseudoLabyrinthe)@\spxentry{construit()}\spxextra{méthode src.utilites.PseudoLabyrinthe}}

\begin{fulllineitems}
\phantomsection\label{\detokenize{src:src.utilites.PseudoLabyrinthe.construit}}
\pysigstartsignatures
\pysiglinewithargsret{\sphinxbfcode{\sphinxupquote{construit}}}{\emph{\DUrole{n}{noeuds}\DUrole{p}{:}\DUrole{w}{  }\DUrole{n}{list}}}{{ $\rightarrow$ None}}
\pysigstopsignatures
\sphinxAtStartPar
Construit un PseudoLabyrinthe à partir d’une liste de Noeuds.
\begin{quote}\begin{description}
\sphinxlineitem{Paramètres}
\sphinxAtStartPar
\sphinxstyleliteralstrong{\sphinxupquote{noeuds}} (\sphinxstyleliteralemphasis{\sphinxupquote{list of Noeud}}) \textendash{} Liste de noeuds à utiliser.

\sphinxlineitem{Lève}
\sphinxAtStartPar
\sphinxstyleliteralstrong{\sphinxupquote{ValueError}} \textendash{} Si le résultat n’est pas un PseudoLabyrinthe valide.

\end{description}\end{quote}

\end{fulllineitems}

\index{copie() (méthode src.utilites.PseudoLabyrinthe)@\spxentry{copie()}\spxextra{méthode src.utilites.PseudoLabyrinthe}}

\begin{fulllineitems}
\phantomsection\label{\detokenize{src:src.utilites.PseudoLabyrinthe.copie}}
\pysigstartsignatures
\pysiglinewithargsret{\sphinxbfcode{\sphinxupquote{copie}}}{}{}
\pysigstopsignatures
\sphinxAtStartPar
Copie le PseudoLabyrinthe courant.
\begin{quote}\begin{description}
\sphinxlineitem{Renvoie}
\sphinxAtStartPar
Copie du PseudoLabyrinthe courant.

\sphinxlineitem{Type renvoyé}
\sphinxAtStartPar
{\hyperref[\detokenize{src:src.utilites.PseudoLabyrinthe}]{\sphinxcrossref{PseudoLabyrinthe}}}

\end{description}\end{quote}

\end{fulllineitems}

\index{get\_noeud\_par\_id() (méthode src.utilites.PseudoLabyrinthe)@\spxentry{get\_noeud\_par\_id()}\spxextra{méthode src.utilites.PseudoLabyrinthe}}

\begin{fulllineitems}
\phantomsection\label{\detokenize{src:src.utilites.PseudoLabyrinthe.get_noeud_par_id}}
\pysigstartsignatures
\pysiglinewithargsret{\sphinxbfcode{\sphinxupquote{get\_noeud\_par\_id}}}{\emph{\DUrole{n}{id}\DUrole{p}{:}\DUrole{w}{  }\DUrole{n}{tuple}}}{}
\pysigstopsignatures
\sphinxAtStartPar
Renvoie le Noeud se trouvant à la position id.
\begin{quote}\begin{description}
\sphinxlineitem{Paramètres}
\sphinxAtStartPar
\sphinxstyleliteralstrong{\sphinxupquote{id}} (\sphinxstyleliteralemphasis{\sphinxupquote{tuple}}) \textendash{} Position du Noeud à chercher

\sphinxlineitem{Renvoie}
\sphinxAtStartPar
Noeud à la position demandée.

\sphinxlineitem{Type renvoyé}
\sphinxAtStartPar
{\hyperref[\detokenize{src:src.utilites.Noeud}]{\sphinxcrossref{Noeud}}}

\sphinxlineitem{Lève}
\sphinxAtStartPar
\sphinxstyleliteralstrong{\sphinxupquote{ValueError}} \textendash{} Si l’id n’est pas valide.

\end{description}\end{quote}

\end{fulllineitems}

\index{get\_noeuds() (méthode src.utilites.PseudoLabyrinthe)@\spxentry{get\_noeuds()}\spxextra{méthode src.utilites.PseudoLabyrinthe}}

\begin{fulllineitems}
\phantomsection\label{\detokenize{src:src.utilites.PseudoLabyrinthe.get_noeuds}}
\pysigstartsignatures
\pysiglinewithargsret{\sphinxbfcode{\sphinxupquote{get\_noeuds}}}{}{}
\pysigstopsignatures
\sphinxAtStartPar
Getter pour la liste de noeuds du PseudoLabyrinthe.
\begin{quote}\begin{description}
\sphinxlineitem{Renvoie}
\sphinxAtStartPar
self.\_\_noeuds

\sphinxlineitem{Type renvoyé}
\sphinxAtStartPar
list of Noeud

\end{description}\end{quote}

\end{fulllineitems}

\index{get\_taille() (méthode src.utilites.PseudoLabyrinthe)@\spxentry{get\_taille()}\spxextra{méthode src.utilites.PseudoLabyrinthe}}

\begin{fulllineitems}
\phantomsection\label{\detokenize{src:src.utilites.PseudoLabyrinthe.get_taille}}
\pysigstartsignatures
\pysiglinewithargsret{\sphinxbfcode{\sphinxupquote{get\_taille}}}{}{}
\pysigstopsignatures
\sphinxAtStartPar
Renvoie la taille du PseudoLabyrinthe.
\begin{quote}\begin{description}
\sphinxlineitem{Renvoie}
\sphinxAtStartPar
self.\_\_taille

\sphinxlineitem{Type renvoyé}
\sphinxAtStartPar
tuple

\end{description}\end{quote}

\end{fulllineitems}

\index{supprime\_connexions\_redoublantes() (méthode src.utilites.PseudoLabyrinthe)@\spxentry{supprime\_connexions\_redoublantes()}\spxextra{méthode src.utilites.PseudoLabyrinthe}}

\begin{fulllineitems}
\phantomsection\label{\detokenize{src:src.utilites.PseudoLabyrinthe.supprime_connexions_redoublantes}}
\pysigstartsignatures
\pysiglinewithargsret{\sphinxbfcode{\sphinxupquote{supprime\_connexions\_redoublantes}}}{}{{ $\rightarrow$ None}}
\pysigstopsignatures
\sphinxAtStartPar
Supprime les connexions rédoublantes dans tous les noeuds du PseudoLabyrinthe courant.

\end{fulllineitems}

\index{supprime\_murs() (méthode src.utilites.PseudoLabyrinthe)@\spxentry{supprime\_murs()}\spxextra{méthode src.utilites.PseudoLabyrinthe}}

\begin{fulllineitems}
\phantomsection\label{\detokenize{src:src.utilites.PseudoLabyrinthe.supprime_murs}}
\pysigstartsignatures
\pysiglinewithargsret{\sphinxbfcode{\sphinxupquote{supprime\_murs}}}{\emph{\DUrole{o}{*}\DUrole{n}{args}}}{{ $\rightarrow$ None}}
\pysigstopsignatures
\sphinxAtStartPar
Procédure qui supprime le mur eventuel entre deux noeuds.
\begin{quote}\begin{description}
\sphinxlineitem{Paramètres}
\sphinxAtStartPar
\sphinxstyleliteralstrong{\sphinxupquote{*args}} (\sphinxstyleliteralemphasis{\sphinxupquote{tuple of Noeud}}) \textendash{} Tuples de Noeuds dont on veut supprimer le mur.

\end{description}\end{quote}

\end{fulllineitems}

\index{verifie() (méthode src.utilites.PseudoLabyrinthe)@\spxentry{verifie()}\spxextra{méthode src.utilites.PseudoLabyrinthe}}

\begin{fulllineitems}
\phantomsection\label{\detokenize{src:src.utilites.PseudoLabyrinthe.verifie}}
\pysigstartsignatures
\pysiglinewithargsret{\sphinxbfcode{\sphinxupquote{verifie}}}{}{}
\pysigstopsignatures
\sphinxAtStartPar
Vérifie que l’objet courant est un PseudoLabyrinthe valide.
\begin{quote}\begin{description}
\sphinxlineitem{Lève}
\sphinxAtStartPar
\sphinxstyleliteralstrong{\sphinxupquote{ValueError}} \textendash{} Si le PseudoLabyrinthe n’est pas valide.

\end{description}\end{quote}

\end{fulllineitems}

\index{verifie\_bidirectionel() (méthode src.utilites.PseudoLabyrinthe)@\spxentry{verifie\_bidirectionel()}\spxextra{méthode src.utilites.PseudoLabyrinthe}}

\begin{fulllineitems}
\phantomsection\label{\detokenize{src:src.utilites.PseudoLabyrinthe.verifie_bidirectionel}}
\pysigstartsignatures
\pysiglinewithargsret{\sphinxbfcode{\sphinxupquote{verifie\_bidirectionel}}}{}{{ $\rightarrow$ None}}
\pysigstopsignatures
\sphinxAtStartPar
Vérifie qu’un PseudoLabyrinthe est bidirectionel.
\begin{quote}\begin{description}
\sphinxlineitem{Lève}
\sphinxAtStartPar
\sphinxstyleliteralstrong{\sphinxupquote{ValueError}} \textendash{} Si le PseudoLabyrinthe courant n’est pas bidirectionel.

\end{description}\end{quote}

\end{fulllineitems}

\index{verifie\_connexions() (méthode src.utilites.PseudoLabyrinthe)@\spxentry{verifie\_connexions()}\spxextra{méthode src.utilites.PseudoLabyrinthe}}

\begin{fulllineitems}
\phantomsection\label{\detokenize{src:src.utilites.PseudoLabyrinthe.verifie_connexions}}
\pysigstartsignatures
\pysiglinewithargsret{\sphinxbfcode{\sphinxupquote{verifie\_connexions}}}{}{{ $\rightarrow$ None}}
\pysigstopsignatures
\sphinxAtStartPar
Vérifie que les connexions entre les noeuds de self.\_\_noeuds sont valides.
\begin{quote}\begin{description}
\sphinxlineitem{Lève}
\sphinxAtStartPar
\sphinxstyleliteralstrong{\sphinxupquote{ValueError}} \textendash{} Si les connexions ne sont pas possibles.

\end{description}\end{quote}

\end{fulllineitems}

\index{verifie\_noeuds() (méthode src.utilites.PseudoLabyrinthe)@\spxentry{verifie\_noeuds()}\spxextra{méthode src.utilites.PseudoLabyrinthe}}

\begin{fulllineitems}
\phantomsection\label{\detokenize{src:src.utilites.PseudoLabyrinthe.verifie_noeuds}}
\pysigstartsignatures
\pysiglinewithargsret{\sphinxbfcode{\sphinxupquote{verifie\_noeuds}}}{}{{ $\rightarrow$ None}}
\pysigstopsignatures
\sphinxAtStartPar
Vérifie que les Noeud de self.\_\_noeuds ont des id valides, et qu’ils remplissent
le PseudoLabyrinthe.
\begin{quote}\begin{description}
\sphinxlineitem{Lève}
\sphinxAtStartPar
\sphinxstyleliteralstrong{\sphinxupquote{ValueError}} \textendash{} Si les conditions ne sont pas satisfaites.

\end{description}\end{quote}

\end{fulllineitems}


\end{fulllineitems}



\subsection{Module contents}
\label{\detokenize{src:module-src}}\label{\detokenize{src:module-contents}}\index{module@\spxentry{module}!src@\spxentry{src}}\index{src@\spxentry{src}!module@\spxentry{module}}

\chapter{Indices and tables}
\label{\detokenize{index:indices-and-tables}}\begin{itemize}
\item {} 
\sphinxAtStartPar
\DUrole{xref,std,std-ref}{genindex}

\item {} 
\sphinxAtStartPar
\DUrole{xref,std,std-ref}{modindex}

\item {} 
\sphinxAtStartPar
\DUrole{xref,std,std-ref}{search}

\end{itemize}


\renewcommand{\indexname}{Index des modules Python}
\begin{sphinxtheindex}
\let\bigletter\sphinxstyleindexlettergroup
\bigletter{s}
\item\relax\sphinxstyleindexentry{src}\sphinxstyleindexpageref{src:\detokenize{module-src}}
\item\relax\sphinxstyleindexentry{src.algorithmes\_docs}\sphinxstyleindexpageref{src:\detokenize{module-src.algorithmes_docs}}
\item\relax\sphinxstyleindexentry{src.utilites}\sphinxstyleindexpageref{src:\detokenize{module-src.utilites}}
\end{sphinxtheindex}

\renewcommand{\indexname}{Index}
\printindex
\end{document}